\documentclass{article}
\usepackage{amsmath}
\usepackage{amssymb}
\title{Math Homework week 2}
\author{Tom Curran}
\begin{document}
\maketitle{}
\section{}

\underline{3.1}

\begin{equation} \label{eq1}
  \begin{split}
  \| x+ y \|^2 & = \langle x+ y, y+x\rangle \\
  & = \langle x,y \rangle + \langle y,x\rangle + \langle x,x\rangle + \langle y,y \rangle \\
  & = \| x\|^2 + \| y \|^2 + \langle y,x\rangle + \overline{\langle x,y \rangle} \\
  & = \| x\|^2 + \| y \|^2 + 2( \langle y,x\rangle )\\
  \langle y,x\rangle = \frac{1}{2}( - \| x\|^2 - \| y \|^2  \| x + y \|^2)
 \end{split}
\end{equation}\label{eq1}

\begin{equation} \label{eq2}
  \begin{split}
  \| x - y \|^2 & = \langle x - y, y - x\rangle \\
  & = \langle x,y \rangle + \langle y,x\rangle - \langle x,x\rangle + \langle y,y \rangle \\
  & = \| x\|^2 + \| y \|^2 -\langle y,x\rangle + \overline{\langle x,y \rangle} \\
  & = \| x\|^2 + \| y \|^2 - 2( \langle y,x\rangle ) \\
  \langle y,x\rangle = \frac{1}{2}(\| x\|^2 + \| y \|^2 - \| x-y \|^2)
 \end{split}
\end{equation}\label{eq2}

combining equations 1 and 2 from above we get

\[ \langle y,x\rangle =  \frac{1}{4} (\| x+ y \|^2 - \| x - y \|^2) \]

\underline{3.2}

\begin{equation} \label{eq3}
  \begin{split}
  \| x + i y \|^2 & = \langle x+ iy, iy+x\rangle \\
  & = \langle x,iy \rangle + \langle iy,x\rangle + \langle x,x\rangle + \langle iy,iy \rangle \\
  & = \| x\|^2 + \| iy \|^2 + \langle iy,x\rangle + \overline{\langle x,iy \rangle} \\
  & = \| x\|^2 + \| iy \|^2 + 2( \langle iy,x\rangle )\\
  \langle y,x\rangle = \frac{1}{2}( - \| x\|^2 - \| iy \|^2  \| x + i y \|^2)
 \end{split}
\end{equation}\label{eq3}

\begin{equation} \label{eq4}
  \begin{split}
  \| x - iy \|^2 & = \langle x - iy, iy - x\rangle \\
  & = \langle x,iy \rangle + \langle iy,x\rangle - \langle x,x\rangle + \langle iy,iy \rangle \\
  & = \| x\|^2 + \| iy \|^2 -\langle iy,x\rangle + \overline{\langle x,iy \rangle} \\
  & = \| x\|^2 + \| iy \|^2 - 2( \langle iy,x\rangle ) \\
  \langle y,x\rangle = \frac{1}{2}(\| x\|^2 + \| iy \|^2 - \| x-iy \|^2)
 \end{split}
\end{equation}\label{eq4}

combining equations 3 and four give:

\[ \langle y,x\rangle =  \frac{1}{4} (i\| x + y \|^2 - i \| x - y \|^2) \]

combining with equations 1 and 2 we get:

\[ \langle y,x\rangle =\frac{1}{4} (\| x+ y \|^2 - \| x - y \|^2  + i\| x + iy \|^2 - i \| x - iy \|^2) \]


\underline{3.3}

\

$\cos\theta =\frac{\langle x,y \rangle}{ \| x\| \|y\|} $

\begin{equation}
  \begin{split}
  \langle f,f \rangle & = \sqrt{\langle f * f \rangle}\\
  & = \sqrt{\int_{0}^{1} f(x)f(x) dx}\\
  & = \sqrt{\int_{0}^{1} x * x dx}\\
  & = \sqrt{\frac{1}{3} - 0}\\
  & = \sqrt{\frac{1}{3}}\\
  & = \| f \|\\
  \end{split}
\end{equation}

\begin{equation}
  \begin{split}
  \langle f,f \rangle & = \sqrt{\langle f * f \rangle}\\
  & = \sqrt{\int_{0}^{1} f(x)f(x) dx}\\
  & = \sqrt{\int_{0}^{1} x^{5} * x^{5} dx}\\
  & = \sqrt{\frac{1}{11} - 0}\\
  & = \sqrt{\frac{1}{11}} \\
  & = \| g \|
  \end{split}
\end{equation}

\begin{equation}
  \begin{split}
  \langle f \cdot g \rangle & = \int_{0}^{1} x * x^5 dx\\
   = \int_{0}^{1}x^6 dx\\
  & = \frac{1}{7}
  \end{split}
\end{equation}

\begin{equation}
  \begin{split}
  \cos \theta = \frac{\frac{1}{7}}{\sqrt{\frac{1}{3}}\sqrt{\frac{1}{11}}}
  \end{split}
\end{equation}

\underline{3.3b}: $x^2, x^4$

$\cos\theta =\frac{\langle x,y \rangle}{ \| x\| \|y\|} $

\begin{equation}
  \begin{split}
  \langle f,f \rangle & = \sqrt{\langle f * f \rangle}\\
  & = \sqrt{\int_{0}^{1} f(x)f(x) dx}\\
  & = \sqrt{\int_{0}^{5} x * x dx}\\
  & = \sqrt{\frac{1}{5} - 0}\\
  & = \sqrt{\frac{1}{5}}\\
  & = \| f \|\\
  \end{split}
\end{equation}

\begin{equation}
  \begin{split}
  \langle f,f \rangle & = \sqrt{\langle f * f \rangle}\\
  & = \sqrt{\int_{0}^{1} f(x)f(x) dx}\\
  & = \sqrt{\int_{0}^{1} x^{4} * x^{4} dx}\\
  & = \sqrt{\frac{1}{17} - 0}\\
  & = \sqrt{\frac{1}{17}} \\
  & = \| g \|
  \end{split}
\end{equation}

\begin{equation}
  \begin{split}
  \langle f \cdot g \rangle & =\int_{0}^{1} x^2 * x^4 dx \\
  & = \int_{0}^{1} x^6 \\
  & = \frac{1}{7}
  \end{split}
\end{equation}

\begin{equation}
  \begin{split}
  \cos \theta = \frac{\frac{1}{7}}{\sqrt{\frac{1}{5}}\sqrt{\frac{1}{17}}}
  \end{split}
\end{equation}

\section{Orthogonality}

\section{QR Decomposition}

\end{document}
