\documentclass{article}
\usepackage{calligra}
\usepackage{amsmath}
\usepackage{amssymb}
\usepackage{amsthm}
\usepackage[utf8]{inputenc}
\usepackage[english]{babel}
\usepackage{amsthm}

\newtheorem{theorem}{Theorem}
\newtheorem*{remark}{Remark}

\theoremstyle{definition}
\newtheorem{definition}{Definition}[section]


\title{Math Homework Week 1}
\author{Tom Curran}
\begin{document}
\maketitle{}
\section{Measure Spaces}

\subsection{Problem 1.3}
\begin{itemize}

\item$\mathcal{G}_1 = \{A, A \subset \mathbb{R}, A \ open \}$

\begin{itemize}
\item if A is an open set on $(-\infty, 0)$ than its complement is $[0, \infty)$
\item According to the definition of a alegebra, A is an $\mathcal{A}$ or $\mathcal{S}$ if it's complement is a closed set. And since $A^c$ is only half closed, $\mathcal{G}_1$ cannot be an algebra
\end{itemize}

\item$\mathcal{G}_2$ = $\{$ A, A is a finite union of intervals of the form $(a,b], (-\infty, b], and (a,\infty) $\}

\begin{itemize}
  \item if $A = \cup_{i=1}^n (a_i, b_i] $ where $-\infty \leq a_i \leq b_i \leq \infty$ then
  \item then $A^c = (-\infty,a_i] \cup \bigcup_{i=1}^{n-1}(b_o,a_{i+1}] \cup (b_n, \infty)) $
  \item given the complement, the function $\mathcal{G}_2$ is an algebra since it goes on indefinitely on both ends and is closed
\end{itemize}
\item$\mathcal{G}_3$ = $\{$ A, A is a countable union of $(a,b], (-\infty, b], and (a,\infty) $\}
  \begin{itemize}

  \item $\mathcal{G}_3$ is a $\sigma$-algebra

  \end{itemize}

\end{itemize}


\subsection{Exercise 1.7}
\
Explain why these are the ’largest’ and ’smallest’ possible $\sigma$-algebras, respectively, in the following sense: if A is any $\sigma$-algebra, then $\{\emptyset, A\} \subset \mathcal{A} \subset \mathcal{P}(X)$.

\begin{itemize}

\item $\{\emptyset, X \}$: this is the smallest algebra $\sigma$-algebra since it contains an empty set and a non-empty set (the empty set's complement
\item $\mathcal{P}(X)$: the Power Set is all subsets of $\mathcal{X}$ including the empty set making $\mathcal{A} \subset \mathcal{P}(X)$ mechanically larger than $\{\emptyset, X\}$

\end{itemize}

\subsection{Exercise 1.10}

\

\underline{Prove the following Proposition}:

\

Let $\{\mathcal{S}_\sigma\}$ be a family of $\sigma$-algebras on X. Then $\cap_\sigma \mathcal{S}_\sigma$ is also a $\sigma$-algebra.

\

Since each set is a $\sigma$-algebra, there must be a set of empty sets, which satisifies the first condition of a an algebra.

\subsection{Exercise 1.17}

\

Let ($X, \mathcal{S}, \mu$) be a measure space. Prove the following:

\begin{itemize}

\item $\mu$ is monotone: if $A, B \in \mathcal{S}$ then $\mu(A) \leq \mu(B)$
\item $\mu$ is countably subadditive: if $\{A_i\}_{i=1}^\infty$ then $\mu(\cup_{i=1}^\infty A_i) \leq \Sigma_{i=1}^\infty \mu(A)$

\end{itemize}

\subsection{Exercise 1.18}

Let ($X, \mathcal{S} \mu$) be a measure space. Let $B \in \mathcal{S}$. Show that $ \lambda \: \mathcal{S} \rightarrow [0, \infty]$ defined by $\lambda(A) = \mu(A \cap B)$ is also a measure of $(X, \mathcal{S})$

\subsection{Exercise 1.20}

\

Prove (ii)

\section{Construction of Lebesgue Measure}

\


\subsection{Exercise 2.10}

\subsection{Exercise 2.14}

\section{Measureable Functions}

\subsection{Exercise 3.1}
\subsection{Exercise 3.4}
\subsection{Exercise 3.7}
\subsection{Exercise 3.14}

\section{Lebesque Integration}
\subsection{Exercise 4.14}

\subsection{Exercise 4.15}

\subsection{Exercise 4.17}

\subsection{Exercise 4.21}
\end{document}
