\documentclass{article}
\usepackage{calligra}
\usepackage{amsmath}
\usepackage{amssymb}
\usepackage{amsthm}
\usepackage[utf8]{inputenc}
\usepackage[english]{babel}
\usepackage{amsthm}

\newtheorem{theorem}{Theorem}
\newtheorem*{remark}{Remark}

\theoremstyle{definition}
\newtheorem{definition}{Definition}[section]


\title{Math Homework Week 1}
\author{Tom Curran}
\begin{document}
\maketitle{}
\section{Problem 1}

\begin{itemize}

\item$\mathcal{G}_1 = \{A, A \subset \mathbb{R}, A \ open \}$

\begin{itemize}
\item if A is an open set on $(-\infty, 0)$ than its complement is $[0, \infty)$
\item According to the definition of a alegebra, A is an $\mathcal{A}$ or $\mathcal{S}$ if it's complement is a closed set. And since $A^c$ is only half closed, $\mathcal{G}_1$ cannot be an algebra
\end{itemize}

\item$\mathcal{G}_2$ = $\{$ A, A is a finite union of intervals of the form $(a,b], (-\infty, b], and (a,\infty) $\}

\begin{itemize}
  \item if $A = \cup_{i=1}^n (a_i, b_i] $ where $-\infty \leq a_i \leq b_i \leq \infty$ then
  \item $A^c = (\infty, )$
\end{itemize}
\item$\mathcal{G}_3$ = $\{$ A, A is a countable union of $(a,b], (-\infty, b], and (a,\infty) $\}

\end{itemize}


\end{document}
